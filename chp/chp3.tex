
\chapter{Perspectiva cónica}

La perspectiva cónica es un sistema de representación gráfico basado en la proyección de un cuerpo tridimensional sobre un plano, mediante rectas proyectantes que pasan por un punto; lugar desde el cual se supone que mira el observador. El resultado final es una representación en el plano de la visión realista obtenida cuando el ojo está en dicho punto, lugar desde el cual aumenta la sensación de estar dentro de la imagen representada.

\section{Elementos}

\begin{figure}[!ht]
  \centering
  \begin{asy}
    size(12cm,0);
    //import markers;
    import geometry;
    //import math;
    path g=scale(1)*unitcircle;
    path gg=scale(1.5)*unitcircle;
    path ggg=scale(2.1)*unitcircle;
    draw(g);
    pair A=(0,0);
    pair V=unit((1,.2));
    pair V1=rotate(90)*(V);
    path l=-2*V--2*V;
    dot(intersectionpoints(l,g));
    pair[] I=intersectionpoints(l,g);

    pair P=(0.3,-0.5), Q=1.5*(V1-A)/2+P;
    pair P1=(length(V)*cos(-2),length(V)*sin(-2)), Q1=4.5*(V1-A)/2+P1;
    pair P2=(2,-1), Q2=4.5*(V1-A)/2+P2;

    dot(midpoint(l)^^V1^^P^^P1^^P2);
    draw(I[0]--I[0]+0.6*(P-I[0])--I[1]^^I[1]--I[1]+0.6*(P-I[1])--I[0], 1*orange+linewidth(0.2mm));

    draw(I[0]--P--I[1]^^Q--I[1]--I[0]--cycle, 1*blue);
    draw(I[0]--P1--I[1]^^Q1--I[1]--I[0]--cycle, 0.8*orange);
    draw(I[0]--P2--I[1]^^Q2--I[1]--I[0]--cycle, 0.8*paleblue);
    line l1=line(-2*V,2*V);
    line l1=line(-2*V,2*V);
    draw(Label("$\mathcal{L}_1$",Relative(.99),align=dir(-45)), l1,
         1bp+dashed+.8red);
    line l2=perpendicular(A,l1);
    draw(Label("$\mathcal{L}_2$",Relative(.99),align=dir(-90)), l2,
         1bp+dashed+0.5*orange);
    line d2=parallel(I[0],l2);
    draw(d2,.8green);
    distance("$www$",offset=10mm,joinpen=dashed,A,I[0],orange);
    draw(I[1]--A,StickIntervalMarker(1,3,size=15,angle=45,blue));
    draw(I[0]--A,StickIntervalMarker(1,3,size=15,angle=45,blue));
    label("$PP$",A,SE,UnFill);
    label("$V$",V1,2*NE,UnFill);
    label("$P_1$",I[0],3*dir(-110),UnFill);
    label("$P_2$",I[1],2*NE,UnFill);

    //show(currentcoordsys);
    perpendicular(A,NE,V1-A,Fill(blue));
    markangle(Label("$\alpha$",Relative(0.5),UnFill),n=3,radius=15,I[1],P,I[0],ArcArrow(5mm,2mm),red);
    markangle(Label("$\alpha_2$",Relative(0.25)),n=3,radius=-5,I[1],Q,I[0],p=0.5blue);

    markangle(Label("$\alpha_1$",Relative(0.5)),n=3,radius=15,I[1],P1,I[0],ArcArrow(5mm,2mm),red);
    markangle(Label("$\gamma_1$",Relative(0.5)),n=3,radius=-15,I[1],Q1,I[0],p=0.5blue);

    markangle(Label("$\alpha_3$",Relative(0.25)),n=3,radius=15,I[1],P2,I[0],ArcArrow(5mm,2mm),red);
    markangle(Label("$\omega$",Relative(0.5)),n=3,radius=-15,I[1],Q2,I[0],p=0.5blue);
  \end{asy}
  \caption{wwwwwwwwwww}
\end{figure}

La perspectiva cónica es un sistema de representación gráfico basado en la proyección de un cuerpo tridimensional sobre un plano, mediante rectas proyectantes que pasan por un punto; lugar desde el cual se supone que mira el observador. El resultado final es una representación en el plano de la visión realista obtenida cuando el ojo está en dicho punto, lugar desde el cual aumenta la sensación de estar dentro de la imagen representada.


\begin{figure}[!ht]
  \centering
  \begin{asy}
size(12cm,0);
//import markers;
import geometry;
//import math;
path g=scale(1)*unitcircle;
path gg=scale(1.5)*unitcircle;
path ggg=scale(2.1)*unitcircle;
draw(g);
pair A=(0,0);
pair V=unit((1,.2));
pair V1=rotate(90)*(V);
pair V2=rotate(-90)*(V);
pair V3=rotate(-90)*(V)+0.9*(V2-V1);
path l=-2*V--2*V;
pair[] I=intersectionpoints(l,g);
dot(I);

pair T=(0.3,-1.2), R=T+0.3*(3,-0.2);
dot(midpoint(l));
dot(V1^^T^^R);

line l1=line(-2*V,2*V);

draw(Label("$\mathcal{L}_1$",Relative(.99),align=dir(-45)), l1,
     1bp+dashed+.8red);
line l2=perpendicular(A,l1);
draw(Label("$\mathcal{L}_2$",Relative(0),align=dir(90)), l2, 1bp+dashed+0.5*orange);
line l3=parallel(V2,l1);draw(l3,green);line l4=parallel(V3,l1);draw(l4,green);
label("$PP$",A,3*NW,UnFill);
label("$V$",V1,W,UnFill);
label("$P_1$",I[0],3*dir(-110),UnFill);
label("$P_2$",I[1],2*NE,UnFill);

transform proj=projection(l4);
point Mp=proj*T;
point Mp1=proj*R;
dot(Mp^^Mp1);
dot(Label("$P_T$",Mp,dir(-90),UnFill),orange);
dot(Label("$P_R$",Mp1,dir(-90),UnFill),orange);
dot(Label("$T$",T,W,UnFill));
dot(Label("$R$",R,N,UnFill));
circle C=circle(Mp,length(Mp-T));
circle C1=circle(Mp1,length(Mp1-R));
draw(C^^C1,dashed);
pair[] II=intersectionpoints(l4,C);
dot(II);
pair[] II1=intersectionpoints(l4,C1);
dot(II1);

transform proj1=projection(l3);
point w1=proj1*II[0], w2=proj1*II[1];
point ww1=proj1*II1[0], ww2=proj1*II1[1];
point T1=proj1*T, R1=proj1*R;
dot(w1^^w2^^ww1^^ww2^^T1^^R1);
dot("$W_1$",w1,dir(90));
dot("$W_2$",w2,dir(-135));
dot("$W_3$",ww1,dir(-120));
dot("$W_4$",ww2,dir(90));
dot("$T_1$",T1,dir(90));
dot("$R_1$",R1,dir(90));
draw(II[0]--w1^^II[1]--w2^^II1[0]--ww1^^II1[1]--ww2^^T--T1^^R--R1, dashed);
draw(I[1]--w1^^I[0]--w2^^I[1]--ww1^^I[0]--ww2^^A--T1^^A--R1, dashed);
draw(T--R, orange+linewidth(0.5mm));
pair[] T3=intersectionpoints(T1--A,w2--I[0]);
pair[] R3=intersectionpoints(R1--A,ww1--I[1]);

dot(R3[0]);
dot(T3[0]);
draw(T3[0]--R3[0], orange+linewidth(0.5mm));
pair R5=rotate(90)*(T-R)+T;
pair RW=rotate(90)*(T-R)+R;
dot("$R_W$",RW,E,UnFill);
dot("$R_5$",R5,W,UnFill);
draw(T--R5--RW--R--cycle, orange+linewidth(0.5mm));
//draw(T--R5, orange+linewidth(0.5mm));
point S1=proj1*R5;
point S2=proj*R5;
dot(S1^^S2);
dot("$S_1$",S1,N);
dot("$S_2$",S2,N);
circle CC=circle(S2,length(R5-S2));
draw(CC,dashed);
pair[] I5=intersectionpoints(l4,CC);
point j=proj1*I5[1], k=proj1*I5[0];
dot(I5);
dot(j);
dot(k);
draw(j--I5[1]^^k--I5[0]^^S1--R5,dashed);
draw(j--I[0]^^k--I[1]^^S1--A,dashed);
pair[] R9=intersectionpoints(S1--A,j--I[0]);
dot(R9[0]);

line l5=line(T3[0],R9[0]);
line l6=line(T3[0],R3[0]);
draw(l5^^l6,dashed+orange);

//line l5=line(T3[0],R9[0]);
line l7=line(intersectionpoint(l5,l1),R3[0]);
line l8=line(intersectionpoint(l6,l1),R9[0]);
draw(T3[0]--R9[0]--intersectionpoint(l7,l8)--R3[0]--cycle, orange+linewidth(0.5mm));
draw(l7^^l8,dashed+orange);
dot("$T_3$",intersectionpoint(l6,l1),N);
dot("$T_8$",intersectionpoint(l7,l1),N);

//show(currentcoordsys);

perpendicular(A,NE,V1-A,Fill(blue));

  \end{asy}
  \caption{La hiperbola $y=\frac{\lVert PP-PD\rVert^2}{x}$}
\end{figure}

La hiperbola $$y=\frac{\lVert PP-PD\rVert^2}{x}$$
La perspectiva cónica es un sistema de representación gráfico basado en la proyección de un cuerpo tridimensional sobre un plano, mediante rectas proyectantes que pasan por un punto; lugar desde el cual se supone que mira el observador. El resultado final es una representación en el plano de la visión realista obtenida cuando el ojo está en dicho punto, lugar desde el cual aumenta la sensación de estar dentro de la imagen representada.

\begin{figure}[!ht]
  \centering
  \begin{asy}
  size(12cm,0);
  import geometry;
  pair A=(0,0);
  pair V=(1,.1);

  pair V1=rotate(90)*(V);
  pair V2=rotate(-90)*(V);
  pair V3=rotate(-90)*(V)+0.9*(V2-V1);

  path g=scale(length(A-V))*unitcircle;
  draw(g);
  path l=-2*V--2*V;
  pair[] I=intersectionpoints(l,g);

  pair T=(0.3,-0.8), f1=A-0.6*(A-V), f2=A+((length(A-V))^2/length(f1-A))*unit(A-V);
  line l1=line(-2*V,2*V);
  line l2=perpendicular(A,l1);
  draw(Label("$\mathcal{L}_1$",Relative(.99),align=dir(-45)), l1, 1bp+dashed+.8red);
  draw(Label("$\mathcal{L}_2$",Relative(0),align=dir(90)), l2, 1bp+dashed+0.5*orange);
  dot("$P_1$",I[0],N);
  dot("$P_2$",I[1],N);
  dot("$F_1$",f1,N);
  dot("$F_2$",f2,N);
  dot("$PP$",A,NW);
  dot("$V$",V1,W);
  dot("$T$",T,dir(-90));
  draw(T--f1^^T--f2);

  pair T1=T+0.5*unit(V1-A);
  pair T2=T+0.2*unit(f1-T);
  dot("$T_1$",T1,dir(90));
  dot("$T_2$",T2,dir(-50));
  pair T3=intersectionpoint(T1--f1,T2--T2+unit(V1-A));
  dot("$T_3$",T3,dir(50));
  pair T4=T+0.4*unit(f2-T);
  dot("$T_4$",T4,dir(-90));
  pair T5=intersectionpoint(T1--f2,T4--T4+unit(V1-A));
  dot("$T_5$",T5,N);
  draw(T1--f1^^T1--f2,blue+dashed);
  draw(T3--f2^^T5--f1^^T2--f2^^T4--f1,orange+dashed);
  draw(T3--T2^^T5--T4^^intersectionpoint(T3--f2,T5--f1)--intersectionpoint(T2--f2,T4--f1)^^T--T1,yellow);
  fill(T2--T3--intersectionpoint(T3--f2,T5--f1)--intersectionpoint(T2--f2,T4--f1)--cycle,blue);
  fill(T4--T5--intersectionpoint(T3--f2,T5--f1)--intersectionpoint(T2--f2,T4--f1)--cycle,orange);
  fill(T3--T2--T--T1--cycle,orange);
  dot("$T_7$",intersectionpoint(T3--f2,T5--f1),N);
  dot("$T_8$",intersectionpoint(T2--f2,T4--f1),N);

  \end{asy}
  \caption{}
\end{figure}

\section{Tipos}
\subsection{Oblicua}
\subsection{Aerea}
\subsection{Frontal}



\section{Sombras}
\subsection{Sombras}
\subsection{Reflejos}
